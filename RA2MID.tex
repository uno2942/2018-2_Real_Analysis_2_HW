\documentclass{article}
\usepackage{graphicx, amssymb}
\usepackage{amsmath}
\usepackage{amsfonts}
\usepackage{amsthm}
\usepackage{kotex}
\usepackage{bm}
\usepackage{hyperref}
\usepackage{xcolor}
\usepackage{mathrsfs}
\usepackage{mathtools}
\usepackage{physics}
\usepackage{ esint }

\textwidth 6.5 truein 
\oddsidemargin 0 truein 
\evensidemargin -0.50 truein 
\topmargin -.5 truein 
\textheight 8.5in

\DeclareMathOperator{\cc}{\mathbb{C}}
\DeclareMathOperator{\rr}{\mathbb{R}}
\DeclareMathOperator{\bA}{\mathbb{A}}
\DeclareMathOperator{\fra}{\mathfrak{a}}
\DeclareMathOperator{\frb}{\mathfrak{b}}
\DeclareMathOperator{\frm}{\mathfrak{m}}
\DeclareMathOperator{\frp}{\mathfrak{p}}
\DeclareMathOperator{\slin}{\mathfrak{sl}}
\DeclareMathOperator{\Lie}{\mathsf{Lie}}
\DeclareMathOperator{\Alg}{\mathsf{Alg}}
\DeclareMathOperator{\Spec}{\mathrm{Spec}}
\DeclareMathOperator{\End}{\mathrm{End}}
\DeclareMathOperator{\rad}{\mathrm{rad}}
\newcommand*\Laplace{\mathop{}\!\mathbin\bigtriangleup}
\newcommand{\id}{\mathrm{id}}
\newcommand{\Hom}{\mathrm{Hom}}
\newcommand{\Sch}{\mathbf{Sch}}
\newcommand{\Ring}{\mathbf{Ring}}
\newcommand{\T}{\mathcal{T}}
\newcommand{\B}{\mathcal{B}}
\newcommand{\Mod}[1]{\ (\mathrm{mod}\ #1)}
\makeatletter
\newcommand*{\rom}[1]{\expandafter\@slowromancap\romannumeral #1@}
\makeatother
\newtheorem{lemma}{Lemma}
\newtheorem{theorem}{Theorem}
\newtheorem{proposition}{Proposition}

\begin{document}

\title{Real Analysis \rom{2} - MID}
\author{SungBin Park, 20150462} 

\maketitle

\section*{Problem 1}
Let $b_i$ be a sequence in $c_0$ such that $1$ in $i$ th position and 0 elsewhere. Let $T\in (c_0)^*$ and $c_i=T(b_i)$. Since $c_0$ have uniform norm, $1=\abs{\sum\limits_{i=1}^N b_i}$ for fixed $N$. Let $\sum\limits_{i=1}^\infty \abs{c_i}=\infty$, and $\norm{T}\leq C$ for some $C>0$. Then there exists $N$ such that $\sum\limits_{i=1}^N \abs{c_i}>C$ and it means $T(\sum\limits_{i=1}^N \frac{\abs{c_i}}{c_i}b_i)>C$ with $\norm{\sum\limits_{i=1}^N \frac{\abs{c_i}}{c_i}b_i}=1$. This is contradiction to boundedness of $T$. Therefore, $(c_i)\in l^1$ converges. Also, this argument shows that $\norm{(c_i)}_{l^1}\leq \norm{T}$. 

Conversely, if $(c_i)\in l^1$, we can make linear functional $T$ by setting $T(b_i)=c_i$ and $T(\sum\limits_{i=1}^\infty \lambda_i b_i)=\sum\limits_{i=1}^\infty \lambda_i c_i$. Then, this is definitely linear, so we only need to check the boundedness. Let $(a_i)\in c_0$ and $\norm{(a_i)}_{c_0}=1$, then $\abs{a_i}\leq 1$ for all $i$, and $T((a_i))=\sum\limits_{i=1}^\infty a_i c_i\leq \sum\limits_{i=1}^\infty \abs{a_ic_i}\leq \sum\limits_{i=1}^\infty \abs{c_i}<\infty$. Therefore, $T$ is bounded and $\norm{T}\leq \norm{(c_i)}_{l^1}$.

Finally, if we let $\phi:(c_0)^*\rightarrow l^1$ by $T\mapsto (c_i)$, $T(b_i)=c_i$, then it is bijective, $\lambda T\mapsto (\lambda c_i)$, $T_1+T_2\mapsto (T_1(b_i)+T_2(b_i))$. Therefore, it is vector space isomorphism. Also, $\norm{T}=\norm{\phi(T)}_{l^1}$ and $\phi$ is isometry isomorphism.

Let $e_i$ be a sequence in $l^1$ such that $1$ in $i$ th position and 0 elsewhere. Let $T\in (l^1)^*$ and $f_i=T(e_i)$ Let $\sup f_i=\infty$, then find a subsequence $(f_{i_j})$ that $f_{i_1}=f_1$ and $f_{i_j}>\max\{\abs{f_{i_{j-1}}}, 2^j\}$. Then, $\sum\limits_{j=1}^\infty \abs{\frac{\abs{f_{i_j}}}{2^jf_{i_j}}e_{i_j}}=1$, but $T\left(\sum\limits_{j=1}^\infty \frac{\abs{f_{i_j}}}{2^j f_{i_j}}e_{i_j}\right))=\sum\limits_{j=1}^\infty \frac{\abs{f_{i_j}}}{2^j}=\infty$. Therefore, $\sup f_i<\infty$ and $(f_i)\in l^\infty$.

Conversely, let $(f_i)\in l^\infty$ such that $\norm{f_i}_{l^\infty}=M$ and make linear functional $T$ from $l^1$ to $l^\infty$ by setting $T(e_i)=f_i$. Linearity is given as before, so I'll prove the boundedness. Let $(c_i)\in l^1$ such that $\norm{(c_i)}_{l^1}=1$. Then, $T((c_i))=\sum\limits_{i=1}^\infty c_i f_i$ and $\norm{T((c_i))}\leq \sum\limits_{i=1}^\infty \abs{c_i f_i}\leq M \sum\limits_{i=1}^\infty \abs{c_i}=M$. Therefore, $T$ is bounded linear functional and $\norm{T}\leq \norm{(f_i)}_{l^\infty}$. Furthermore, for any $\epsilon>0$, there exists $j$ such that $\abs{f_j}\geq M-\epsilon$ and $T\left(\frac{\abs{f_j}}{f_j}e_j\right)\geq M-\epsilon$. This is true for any $\epsilon$, so $\norm{T}\geq M$. Therefore, $\norm{T}=\norm{(f_i)}_{l^\infty}$.

Finally, if we let $\phi:(l^1)^*\rightarrow l^\infty$ by $T\mapsto (f_i)$, $T(e_i)=f_i$, then it is bijective, $\lambda T\mapsto (\lambda f_i)$, $T_1+T_2\mapsto (T_1(e_i)+T_2(e_i))$. Therefore, it is vector space isomorphism. Also, $\norm{T}=\norm{\phi(T)}_{l^\infty}$ and $\phi$ is isometry isomorphism.

\section*{Problem 2}
\begin{enumerate}
\item[a.] Since $f\in L^2(\mathbb{T})$, $\hat{f}(k)=\int_0^1 \left(\frac{1}{2}-x\right) e^{-2\pi ik x} dx=\frac{1 i}{2\pi n}$ for $k\neq 0$ If $k=0$, $\hat{f}(0)=\int_0^1 \left(\frac{1}{2}-x\right) dx = 0$.
\item[b.] Using Parseval's identity, $\norm{f}_{L^2(\mathbb{T})}=\sum\limits_{k=-\infty}^{k=\infty} \abs{\hat{f}(k)}^2$. $\frac{1}{12}=2\sum\limits_{n=1}^{n=\infty} \frac{1}{4\pi^2 n^2} = \sum\limits_{n=1}^{n=\infty} \frac{1}{2\pi^2 n^2}$. Therefore,
\begin{equation*}
\frac{\pi^2}{6}=\sum\limits_{n=1}^{n=\infty} \frac{1}{n^2}
\end{equation*}
\end{enumerate}
\section*{Problem 3}
Since $f$ is continuous on compact $\mathbb{T}$, $f\in L^1(\mathbb{T})$. Therefore, $f* K_n$ exists for almost $x$ and is in $L^1$. Therefore, small enough $\epsilon>0$, there exists $\delta$ such that $\abs{f(x)-f(y)}\leq \epsilon$ when $\abs{x-y}\leq \delta$. Also, there exists $N$ such that for all $n\geq N$,
\begin{equation*}
f * K_n(x)=\int_0^1 f(x-y)K_n(y) = \left(\int_0^\delta + \int_\delta^{1-\delta} + \int_{1-\delta}^1\right) f(x-y)K_n(y) dy=  \left(\int_0^\delta + \int_{1-\delta}^1\right) f(x-y)K_n(y) dy+\epsilon/3
\end{equation*}
The last term can be rewritten by
\begin{equation*}
\left(\int_0^\delta + \int_{1-\delta}^1\right) f(x-y)K_n(y) dy = \int_{x-\delta}^x f(y) K_n(x-y) dy +\int_{x-1}^{x-1+\delta} f(y) K_n(x-y)dy = \int_{x-\delta}^{x+\delta} f(y) K_n(x-y) dy
\end{equation*}

Since $\int_0^1 K_n(y)dy=1$, $\int_{-\delta}^{\delta} K_n(y)\geq 1-\epsilon$.

\begin{equation*}
\int_{x-\delta}^{x+\delta} (f(y)-f(x)) K_n(x-y) dy + \int_{x-\delta}^{x+\delta}f(x)K_n(x-y) dy \leq \epsilon \int_{x-\delta}^{x+\delta} K_n(x-y) dy + f(x)(1-\epsilon) \leq \epsilon M + f(x)(1-\epsilon).
\end{equation*}
Therefore, for fixed $\epsilon>0$, there exists $N$ such that for all $n\geq N$,
\begin{equation*}
\abs{f*K_n(x)-f(x)}\leq \epsilon\left(\frac{1}{3}+M-f(x)\right)\leq  \epsilon\left(\frac{1}{3}+M+\norm{f(x)}_{C^0}\right)
\end{equation*}
for all $x\in \mathbb{T}$. It means $f*K_n(x)\rightarrow f(x)$ uniformly.
\section*{Problem 4}
\begin{enumerate}
\item[(1)] For convenience, assume $e^{2\pi i x}\neq 1$.
\begin{equation*}
\begin{split}
NF_N(x)+N&=2\sum\limits_{k=0}^{N-1} (N-k)e^{2\pi i k x} = 2N\left(\frac{1-e^{2\pi i N x}}{1-e^{2\pi i x}}\right) - \frac{2}{2\pi i }\pdv{x} \sum\limits_{k=0}^{N-1}e^{2\pi i k x} \\
&=2N\left(\frac{1-e^{2\pi i N x}}{1-e^{2\pi i x}}\right)- \frac{2}{2\pi i }\pdv{x}\left(\frac{1-e^{2\pi i N x}}{1-e^{2\pi i x}}\right) = \\
&2N\left(\frac{1-e^{2\pi i N x}}{1-e^{2\pi i x}}\right) - 2\frac{(1-e^{2\pi i N x})(e^{2\pi i x}) - Ne^{2\pi i N x}(1-e^{2\pi i x})}{(1-e^{2\pi i x})^2} \\
&=2Ne^{\pi i (N-1)x} \frac{\sin{\pi N x}}{\sin{ \pi x}} + 2\frac{e^{\pi i (N-1)x}}{2i}\frac{\sin{(\pi N x)}e^{\pi i x}- Ne^{\pi i N x}\sin{(\pi x)}}{\sin^2{\pi x}}
\end{split}
\end{equation*}
Since $\overline{F_N}(x)=F_N(x)$, it is real valued function, so
\begin{equation*}
\begin{split}
NF_N(x)+N&=2N\cos{(N-1)\pi x}\frac{\sin{\pi N x}}{\sin{ \pi x}} + \frac{\left(\sin(\pi N x)\right)^2 -N \sin(\pi (2N-1) x)\sin(\pi x)}{\sin^2(\pi x)} \\
&=\frac{\sin^2(\pi N x)}{\sin^2(\pi x)} + N\frac{\cos((N-1)\pi x) \sin (N \pi x)- \sin((N-1)\pi x) \cos (N \pi x)}{\sin(\pi x)} = \frac{\sin^2(\pi N x)}{\sin^2(\pi x)} + N
\end{split}
\end{equation*}
Therefore, $F_N(x)=\frac{1}{N}\frac{\sin^2(\pi N x)}{\sin^2(\pi x)}$ for $e^{2\pi i x}\neq 1$. If $e^{2\pi i x}=1$, then $NF_N(x)+N=N(N+1)$, so $F_N(x)=N$. Since $\lim\limits_{x\rightarrow n}F_N(x) = \frac{1}{N}N^2=N$, we can get
\begin{equation*}
F_N(x)=\frac{1}{N}\frac{\sin^2(N\pi x/2)}{\sin^2{(\pi x/2)}}
\end{equation*}
\item[(2)] First, $F_N(x+1)=F_N(x)$ for all $x$ since $e^{2\pi i k x}$ is periodic function with period $1$. Therefore, $F_N(x)$ is on $\mathbb{T}$.
\begin{enumerate}
\item[(i)] For $k\neq 0$,
\begin{equation*}
\int_0^1 e^{2\pi i k x} dx=0,
\end{equation*}
so $\int_0^1 F_N(x) dx = \frac{1}{N}\int_0^1 N dx = N$.
\item[(ii)] Since $F_N>0$, $\int_0^1 F_N(x) dx = 1 < 2$.
\item[(iii)] For $\delta>0$, there exists $N>0$ such that $\frac{1}{N}< \delta$. For the $N$,
\begin{equation*}
\int_\delta^{1-\delta} \abs{F_{N^2}(x)} dx \leq \sum\limits_{k=N}^{N^2-N}\frac{1}{N^2}\frac{1}{\sin^2(\frac{2k+1}{4N^2}\pi)}\frac{1}{N} \leq \sum\limits_{k=N}^{N^2-N}\frac{1}{N^2}\frac{2N^2}{2k+1}\frac{1}{N} \leq \frac{1}{N}\ln\left(\frac{2(N^2-N+1)+1}{2N+1}\right)
\end{equation*}
Therefore, $\lim\limits_{n\rightarrow \infty}\int_\delta^{1-\delta}\abs{F_n(x)}dx=0$
\end{enumerate}
\end{enumerate}
\section*{Problem 5}
\begin{enumerate}
\item[(1)] $f*K_n(x)=\int_0^1 f(x-y)K_n(y)dy=\sum\limits_{k=-N+1}^{N-1} \int_0^1 kf(x-y)e^{2 \pi i k y}dy = \sum\limits_{k=-N+1}^{N-1} ke^{2\pi i k x}\int_{x-1}^x f(y)e^{-2\pi ik y}=\sum\limits_{k=-N+1}^{N-1} ke^{2\pi i k x}\hat{f}(k)=0$ for all $n\in \mathbb{Z}$ and $x\in \mathbb{T}$, and by problem 3, $f*K_n\rightarrow 0$ uniformly, therefore, $f=0$.
\item[(2)] Let $f$ be a continuous function on $\mathbb{T}$ and fix $\epsilon>0$. Then, there exists $N$ such that $\norm{f*F_n-f}_u\leq \epsilon$ for all $n\geq N$. However, $f*F_n(x)=\int_0^1 f(x-y)F_n(y)dy = \int_{x-1}^x f(y)KF_n(x-y)dy =  \left(\int_{x-1}^0 + \int_0^x\right) f(y)F_n(x-y)dy =\int_0^1 f(y)F_n(x-y)dy=\sum\limits_{k=-N+1}^{N-1} ke^{2\pi i k x}\int_0^1 f(y)e^{-2\pi i k y}dy$ and this is trigonometric polynomial since $f(y)e^{-2\pi i k y}$ is continuous on compact $\mathbb{T}$ for all $k$. Therefore, continuous functions on $\mathbb{T}$ can be uniformly approximated by trigonometric polynomials.
\end{enumerate}
\section*{Problem 6}
$\int_{\rr} f(y)e^{\pi(-y^2+2xy)}dy=0 \Leftrightarrow \int_{\rr} f(y)e^{-\pi(x-y)^2}dy= f*g(x)=0$ for $g(x)=e^{-\pi x^2}$. Let $\phi(\xi)=\exp(2\pi i \xi x-\pi \xi^2)$, then
\begin{equation*}
\hat{\phi}(y)=e^{-\pi x^2}\left(e^{-\pi(\xi-ix)^2)}\right)^\wedge=e^{-\pi x^2}\left(\tau_{ix}e^{-\pi\xi^2}\right)^\wedge = e^{-\pi(x^2-2yx+y^2)}
\end{equation*}
Therefore, $\hat{\phi}(x-y)=g(y)$ and $f*g(x)=\int_{\rr}f\hat{\phi}=\int_{\rr}\hat{f}\phi=\int_{\rr}\hat{f}(\xi)e^{2\pi i \xi x-\pi \xi^2}d\xi$. Let $h(\xi)=\hat{f}(\xi)e^{-\pi\xi^2}$, then $\int_{\rr}h(\xi)e^{2\pi i \xi x}=0$ for all $x$. It means $\hat{h}(x)=0$ for all $x$. Since $h\in L^1$ and $\hat{h}=0$, $h=0$ a.e. and it means $\hat{f}(\xi)=0$ a.e. Since $f\in L^1$ and $\hat{f}=0$, $f=0$ a.e. and $f\equiv 0$ since $f\in C(\mathbb{R})$.
\end{document}