\documentclass{article}
\usepackage{graphicx, amssymb}
\usepackage{amsmath}
\usepackage{amsfonts}
\usepackage{amsthm}
\usepackage{kotex}
\usepackage{bm}
\usepackage{hyperref}
\usepackage{xcolor}
\usepackage{mathrsfs}
\usepackage{mathtools}
\usepackage{physics}
\usepackage{ esint }

\textwidth 6.5 truein 
\oddsidemargin 0 truein 
\evensidemargin -0.50 truein 
\topmargin -.5 truein 
\textheight 8.5in

\DeclareMathOperator{\cc}{\mathbb{C}}
\DeclareMathOperator{\rr}{\mathbb{R}}
\DeclareMathOperator{\bA}{\mathbb{A}}
\DeclareMathOperator{\fra}{\mathfrak{a}}
\DeclareMathOperator{\frb}{\mathfrak{b}}
\DeclareMathOperator{\frm}{\mathfrak{m}}
\DeclareMathOperator{\frp}{\mathfrak{p}}
\DeclareMathOperator{\slin}{\mathfrak{sl}}
\DeclareMathOperator{\Lie}{\mathsf{Lie}}
\DeclareMathOperator{\Alg}{\mathsf{Alg}}
\DeclareMathOperator{\Spec}{\mathrm{Spec}}
\DeclareMathOperator{\End}{\mathrm{End}}
\DeclareMathOperator{\rad}{\mathrm{rad}}
\newcommand*\Laplace{\mathop{}\!\mathbin\bigtriangleup}
\newcommand{\id}{\mathrm{id}}
\newcommand{\Hom}{\mathrm{Hom}}
\newcommand{\Sch}{\mathbf{Sch}}
\newcommand{\Ring}{\mathbf{Ring}}
\newcommand{\T}{\mathcal{T}}
\newcommand{\B}{\mathcal{B}}
\newcommand{\Mod}[1]{\ (\mathrm{mod}\ #1)}
\makeatletter
\newcommand*{\rom}[1]{\expandafter\@slowromancap\romannumeral #1@}
\makeatother
\newtheorem{lemma}{Lemma}
\newtheorem{theorem}{Theorem}
\newtheorem{proposition}{Proposition}

\begin{document}

\title{Real Analysis \rom{2} - FINAL}
\author{SungBin Park, 20150462} 

\maketitle

\section*{Problem 1}
Let $\mu\in M(\rr^n)$. Construct $\phi_t\in C_c^\infty(\rr^n)$ for $t>0$ such that $\int \phi_t dx=1$. More explicitly, we can set $\phi_t(x)=t^{-n}\phi(t^{-1}x)$ where $\phi(x)$ defined as
\begin{equation*}
\phi(x)=\begin{cases}
\exp[(\abs{x}^2-1)^{-1}] & \text{if }\abs{x}<1 \\
0 & \text{if }\abs{x}\geq 1.
\end{cases}
\end{equation*}
Since $\phi_t\in L^1(\rr^n)$ for all $t>0$, $\phi_t*\mu(x)=\int f(x-y)d\mu(y)$ exists for a.e. x, $\phi_t*\mu\in L^1$, and $\norm{\phi_t * \mu}_1\leq \norm{\mu}=\abs{\mu}(\rr^n)$. Since $\phi_t*\mu\in L^1$, we can identify $\phi_t*\mu\in L^1$ as a Radon measure on $\rr^n$ and denote $d\mu_t=(\phi_t*\mu)dm$. I'll show that $\mu_t\rightarrow\mu$ vaguely by showing that $\int fd\mu_t\rightarrow\int fd\mu$ for all $f\in C_0(\rr^n)$.

Fix $f\in C_0(\rr^n)$, then
\begin{equation*}
\begin{split}
\int_{\rr^n} fd\mu_t = \int_{\rr^n} f (\phi_t*\mu)dm&=\int_{\rr^n} f(x) \left(\int_{\rr^n} \phi_t(x-y)d\mu(y)\right)dx \\
&=\int_{\rr^n}\int_{\rr^n} f(x)\phi_t(x-y)dxd\mu(y)~(\text{By Fubini theorem}) \\
&=\int_{\rr^n} (f*\tilde{\phi_t})(y)d\mu(y)
\end{split}
\end{equation*}
The Fubini theorem can be applied since $f(x)\phi_t(x-y)$ is continuous on $\rr^n\times \rr^n$ and $\abs{\int_{\rr^n} f (\phi_t*\mu)dm}\leq \norm{f}_\infty \int_{\rr^n}\abs{(\phi_t*\mu)}dm<\infty$, implying $f(x)\phi_t(x-y)\in L^1(\mu\times m)$. (Since the function is continuous, we can consider $m$ be a Borel measure restricting to $\mathcal{B}_{\rr^n}$.)

I claim that $f*\tilde{\phi_t}\in C_0(\rr^n)$. Since $f\in C_0(\rr^n)$, there exists a sequence $\{f_j\}\subset C_c(\rr^n)$ such that $\norm{f_j-f}_u\rightarrow 0$ as $j\rightarrow \infty$. Note that $f\in L^\infty$ and $\phi_t\in L^1$, so $f*\phi_t$ is bounded and uniform continuous. As $f_j,\tilde{\phi_t}\in C_c$, $f_j*\tilde{\phi_t}\in C_c$ and
\begin{equation*}
\norm{f_j*\tilde{\phi_t}-f*\tilde{\phi_t}}_u=\norm{(f_j-f)*\tilde{\phi_t}}_u\leq \norm{(f_j-f)}_\infty\norm{\phi_t}_1=\norm{(f_j-f)}_u\norm{\phi_t}_1\rightarrow 0
\end{equation*}
Therefore, $f*\tilde{\phi_t}\in C_0(\rr^n)$.

Since $f\in C_0(\rr^n)$, it is uniformly continuous, (Let $U\subset \rr^n$ such that $f(x)\leq \epsilon$, then $U^c$ is bounded as $f\in C_0$, so $U^c$ is compact. Therefore, $\abs{f(x)-f(y)}$ can be controlled by controlling $\abs{x-y}$ uniformly.) and $f*\tilde{\phi_t}\rightarrow f$ uniformly as $t\rightarrow 0$. As $\mu\in M(\rr^n)$, $f\mapsto \int fd\mu$ is bounded linear functional, and $\int f*\tilde{\phi_t} d\mu\rightarrow \int f d\mu$ as $t\rightarrow \infty$. It means
\begin{equation*}
\int_{\rr^n} fd\mu_t=\int_{\rr^n} (f*\tilde{\phi_t})(y)d\mu(y)\rightarrow \int_{\rr^n} fd\mu
\end{equation*}
as $t\rightarrow \infty$ and $\mu_t\rightarrow\mu$. Since $\mu_t$ is generated by $L^1$ functions, this result shows that $L^1(\rr^n)$ is dense in $M(\rr^n)$.
\section*{Problem 2}
\begin{enumerate}
\item[(1)] By the fundamental Theorem of Calculus for Lebesgue Integrals, $F$ is absolutely continuous on $[0, 1]$. Also, $F$ is bounded variation $[0,1]$. I need to show that $F(0)=F(1)$ to state that $F\in C(\mathbb{T})$, but $\hat{f}(0)=0$ implies
\begin{equation*}
\hat{f}(0)=\int_0^1 f(x)dx=0.
\end{equation*}
Therefore, $F(1)=0=F(0)$ and $F\in C(\mathbb{T})$.

By the fundamental Theorem of Calculus for Lebesgue Integrals again, $F$ is differentiable a.e. on $[0,1]$ and $F'=f$. Therefore,
\begin{equation*}
\begin{split}
\hat{F}(n)=\int_0^1 F(x)e^{-2\pi i n x} dx&=\left[\frac{1}{-2\pi i n} F(x)e^{-2\pi i n x}\right]_0^1 + \frac{1}{2\pi i n}\int_0^1 f(x)e^{-2\pi i n x} dx \\
&=\frac{1}{2\pi i n}f(n)
\end{split}
\end{equation*}
for $n\neq 0$ and if $n=0$, $\hat{F}(0)<\infty$ since $F$ is continuous on $[0,1]$, which is compact set.
\item[(2)] Since $F$ is bounded variation on $\mathbb{T}$, we can apply Fej\'er's theorem and get
\begin{equation*}
\lim\limits_{N\rightarrow\infty}\frac{1}{N}\sum\limits_{n=0}^{N-1}(S_n F)(0)=F(0).
\end{equation*}
By the definition of $(S_n F)(0)$,
\begin{equation*}
\begin{split}
\lim\limits_{N\rightarrow\infty}\frac{1}{N}\sum\limits_{n=0}^{N-1}\sum\limits_{k=-n}^n \hat{F}(k)&=\lim\limits_{N\rightarrow\infty}\frac{1}{N}\left(N\hat{F}(0)+\sum\limits_{n=1}^{N-1}(N-n)(\hat{F}(n)+\hat{F}(-n))\right) \\
&=\hat{F}(0)+\lim\limits_{N\rightarrow\infty}2\sum\limits_{n=1}^{N-1}\left(1-\frac{n}{N}\right)\hat{F}(n) \\
&=\hat{F}(0)+\lim\limits_{N\rightarrow\infty}2\sum\limits_{n=1}^{N-1}\left(1-\frac{n}{N}\right)\frac{\hat{f}(n)}{2i\pi n} \\
&=\hat{F}(0).
\end{split}
\end{equation*}
(In the calculation, I used $\hat{F}(-n)=\frac{1}{-2\pi i n}\hat{f}(-n)=\frac{1}{2\pi i n}\hat{f}(n)=\hat{F}(n)$.) Therefore,
\begin{equation*}
\lim\limits_{N\rightarrow\infty}2\sum\limits_{n=1}^{N-1}\left(1-\frac{n}{N}\right)\frac{\hat{f}(n)}{n}=2\pi i(F(0)-\hat{F}(0))=-2\pi i \hat{F}(0)
\end{equation*}
since $F(0)=0$.
\item[(3)] By Hausdorff-Young Inequality, if $f\in L^1(\mathbb{T})$, $ \hat{f}\in l^\infty(\mathbb{Z})$ and $\norm{\hat{f}}_\infty\leq \norm{f}_1$. Therefore, 
\begin{equation*}
\lim\limits_{N\rightarrow\infty}2\sum\limits_{n=1}^{N-1}\left(\frac{\hat{f}(n)}{N}\right)\leq \norm{f}_1
\end{equation*}
and
\begin{equation*}
\norm{\lim\limits_{n\rightarrow \infty} \frac{1}{n}\hat{f}(n)}\leq \norm{\norm{f}_1}+\norm{\pi i \hat{F}(0)}<\infty
\end{equation*}
for the norm on $\mathbb{C}$, and $\lim\limits_{n\rightarrow\infty} \frac{1}{n}\hat{f}(n)< \infty$ since $\hat{f}(n)\in \rr$ for all $n\in \mathbb{Z}$.
\item[(4)] By integral test of $\sum\limits_{n=2}^\infty \frac{1}{n\log n}$,
\begin{equation*}
\sum\limits_{n=2}^\infty \frac{1}{n\log n}\leq \frac{1}{2\log 2}+ \int_2^\infty \frac{1}{x\log x}dx=\frac{1}{2\log 2}+ \left[\ln t\right]_{\ln 2}^\infty = \infty, 
\end{equation*} 
so $\sum\limits_{n=2}^\infty \frac{1}{n\log n}\rightarrow \infty$ and it is not a Fourier series of $L^1$ function. However,
\begin{equation*}
\sum\limits_{n=2}^\infty \frac{\sin 2\pi n x}{\log n}
\end{equation*}
converges for all $x$: computing $\sum\limits_{i=2}^N e^{2\pi i n x}$, it is $ \frac{2\pi i (N+1)x-e^{4\pi i x}}{-1+e^{2\pi i x}}$, whose absolute value is upper bounded by $\frac{2}{-1+e^{2\pi i x}}$ for fixed $x$ without $x=0,1$ for all $N$. Therefore, $\sum\limits_{i=2}^N \sin(2\pi i n x)$ is bounded by $\abs{M(x)}<\infty$ for fixed $x$ in $[0, 1]$ for all $N$. Since $\frac{1}{\log n}$ is decreasing sequence, by Dirichlet test, $\sum\limits_{n=2}^\infty \frac{\sin 2\pi n x}{\log n}$ converges for all $x$.
\end{enumerate}
\section*{Problem 3}
I'll follow the steps in exercise 11 in sec 9.1.

\begin{enumerate}
\item[(1)] There exist $N\in \mathbb{N}$, $C>0$ such that for all $\phi\in C_c^\infty$,
\begin{equation}
\abs{\langle F,\phi\rangle}\leq \sum\limits_{\abs{\alpha}\leq N}\sup\limits_{\abs{x}\leq 1}\abs{\partial^\alpha \phi(x)}.
\end{equation}
\begin{proof}
$F:C^\infty_c(\rr^n)\rightarrow\rr$, which is continuous linear functional, and $C^\infty_c(\rr^n)$ is Fr\'echet space with the topology defined by the norms
\begin{equation*}
\phi\rightarrow \norm{\partial^\alpha \phi}_u~~~(\alpha\in\{0,1,2,\ldots\}^n)
\end{equation*}
for $K\subset U$, $\phi\in C_c^\infty (K)$. Therefore, by proposition 5.15 in Folland, there exists $\alpha_1,\ldots, \alpha_k\in\{0,1,2,\ldots\}^n$ and $C>0$ such that
\begin{equation*}
\abs{\langle F,\phi\rangle}\leq C\sum\limits_{j=1}^k \norm{\partial^{\alpha_j}\phi}_u
\end{equation*}
To concentrate only on $\abs{x}\leq 1$ region, make $C_c^\infty$ function $\varphi$ using $C^\infty$ Urysohn lemma such that $\varphi=1$ on $\abs{x}\leq 1/2$ and $\varphi=0$ at $\abs{x}\geq 1$. Denote $\phi_1=\varphi\phi$ and $\phi_2=(1-\varphi)\phi$. Then, $\phi_1+\phi_2=\phi$ and
\begin{equation*}
\abs{\langle F,\phi\rangle}=\abs{\langle F,\phi_1\rangle+\langle F,\phi_2\rangle}=\abs{\langle F,\phi_1\rangle}
\end{equation*}
since $\text{supp}(F)=\{0\}$. Hence,
\begin{equation*}
\abs{\langle F,\phi\rangle}=\abs{\langle F,\phi_1\rangle}\leq C\sum\limits_{\abs{\alpha}\leq N}\sup\limits_{\abs{x}\leq 1}\abs{\partial^\alpha\phi(x)}
\end{equation*}
where $N=\abs{\alpha_k}$.
\end{proof}
\item[(2)]Take $\psi\in C_c^\infty$ with $\psi(x)=1$ for $\abs{x}\leq 1$ and $\psi(x)=0$ for $\abs{x}\geq 2$. Assume $\phi\in C_c^\infty$ and $\partial^\alpha\phi(0)=0$ for $\abs{\alpha}\leq N$. Let $\phi_k(x)=\phi(x)(1-\psi(kx))$, then $\partial^\alpha_k\phi\rightarrow\partial^\alpha\phi$ uniformly as $k\rightarrow \infty$ for $\abs{\alpha}\leq N$.
\begin{proof}
Since $\partial^\alpha\phi_k(x)$ is compactly supported on the support of $\phi$ for $\abs{\alpha}\leq N$, we just need to show that $\partial^\alpha\phi_k\rightarrow\partial^\alpha \phi$. Fix $\alpha$ such that $\abs{\alpha}\leq N$, then
\begin{equation*}
\abs{\partial^\alpha\phi-\partial^\alpha\phi_k}=\abs{\partial^\alpha(\phi-\phi_k)}=\abs{\partial^\alpha(\phi(x)\psi(kx))}
\end{equation*}
and $\partial^\alpha(\psi(kx))=0$ on $\abs{x}> 2/k$. Using Leibniz rule and hint, $\abs{\partial^\alpha \phi(x)}\leq C\abs{x}^{N+1-\abs{\alpha}}$ for $\abs{\alpha}\leq N$ on $\abs{x}\leq 1$, which will be proven later,
\begin{equation*}
\begin{split}
\abs{\partial^\alpha(\phi(x)\psi(kx))}&=\abs{\sum\limits_{\beta+\gamma=\alpha}\frac{\alpha!}{\beta!\gamma!}(\partial_x^\beta (\phi(x)))(\partial_x^\gamma (\psi(kx)))} \\
&\leq \sum\limits_{\beta+\gamma=\alpha}\frac{\alpha!}{\beta!\gamma!}\abs{(\partial_x^\beta (\phi(x)))(\partial_x^\gamma (\psi(kx)))} \\
&\leq C\sum\limits_{\beta+\gamma=\alpha}\frac{\alpha!}{\beta!\gamma!}\abs{x}^{N+1-\abs{\beta}} \abs{k}^\abs{\gamma}\abs{(\partial^\gamma\psi)(kx)} \\
&=C\sum\limits_{\beta+\gamma=\alpha}\frac{\alpha!}{\beta!\gamma!}\abs{kx}^{\abs{\gamma}} \abs{x}^\abs{N+1-\abs{\beta}-\abs{\gamma})}\abs{(\partial^\gamma\psi)(kx)}
\end{split}
\end{equation*}
in the support of $\psi(kx)\subset \{\abs{x}\leq 1\}$. The support of $\psi$ is a subset of $\abs{kx}\leq 2$ and $\abs{N+1-\abs{\beta}-\abs{\gamma})}=\abs{N+1-\abs{\alpha}}\geq 1$ for $\abs{\alpha}\leq N$. Also, $\partial^\gamma\psi(x)$ is bounded for $\abs{\gamma}\leq N$ since it is $C_c^\infty$ function. Summarising the fact,
\begin{equation*}
\abs{\partial^\alpha(\phi(x)\psi(kx))}\leq C'\abs{x}\leq \frac{2C'}{k}
\end{equation*}
for some $C'>0$ for all $\abs{\alpha}\leq N$ and goes to $0$ as $k\rightarrow\infty$. Therefore,
\begin{equation*}
\partial^\alpha\phi_k\rightarrow\partial^\alpha\phi
\end{equation*}
uniformly as $k\rightarrow \infty$ for $\abs{\alpha}\leq N$.

pf of hint: The Taylor series of $\phi$ about $0$ would be
\begin{equation*}
\phi(x)=\sum_{\abs{\beta}\leq N+1} \frac{\partial^\beta \phi(0)}{\beta!} x^\beta + r(x)
\end{equation*}
such that $r(x)$ is differentiable and $\lim\limits_{x\rightarrow 0}\frac{r(x)}{\abs{x}^{N+1}}=0$. Since we only interested in $\abs{x}\leq 1$, $\partial^\beta\phi$ is bounded for all $\abs{\beta}\leq N+1$ and and $\frac{r(x)}{x^{N+1}}$ is bounded in $\abs{x}\leq 1$. As $\partial^\beta \phi(0)=0$ for $\abs{\beta}\leq N$, we can get
\begin{equation*}
\abs{\phi(x)}\leq \left(C+\frac{r(x)}{\abs{x}^{N+1}}\right) \abs{x}^{N+1}\leq C' \abs{x}^{N+1}.
\end{equation*}
for some $0<C, C'<\infty$. By the same reason, we can get
\begin{equation*}
\abs{\partial^\alpha \phi(x)}\leq C\abs{x}^{N+1-\abs{\alpha}}
\end{equation*}
for $\abs{\alpha}\leq N$.
\end{proof}
\item[(3)] Considering $\langle F,\phi-\phi_k\rangle$,
\begin{equation*}
\begin{split}
\abs{\langle F,\phi-\phi_k\rangle}\leq C\sum\limits_{\abs{\alpha}\leq N}\sup\limits_{\abs{x}\leq 1}\abs{\partial^\alpha(\phi-\phi_k)(x)}
\end{split}
\end{equation*}
and we already showed that $\abs{\partial^\alpha(\phi-\phi_k)(x)}\rightarrow 0$ uniformly as $k\rightarrow\infty$ for $\abs{\alpha}\leq N$. Therefore,
\begin{equation*}
\abs{\langle F,\phi\rangle}=\lim\limits_{k\rightarrow \infty} \abs{\langle F,\phi_k\rangle}.
\end{equation*}
For all $k$, $\langle F, \phi_k\rangle=0$ since $\phi_k(x)=0$ on $\abs{x}\leq 1/k$. Therefore, $\langle F, \phi \rangle=0$ if $\phi\in C_c^\infty$ and $\partial^\alpha\phi(0)=0$ for $\abs{\alpha}\leq N$.
\item[(4)] I'll use $\varphi\in C_c^\infty$ in step 1 again. Assume $\phi\in C_c^\infty$ and $\partial^\alpha \phi(0)=a_\alpha$, then let $f(x)=\varphi\sum\limits_{\abs{\alpha}\leq N} \frac{a_\alpha}{\alpha!} x^\alpha $. Then, $f\in C_c^\infty$ and $g=\phi-f\in C_c^\infty$ such that $\partial^\alpha g(0)=0$ for all $\abs{\alpha}\leq N$. Therefore $\langle F, g\rangle=0=\langle F, \phi\rangle-\langle F, f\rangle$, and $\langle F, \phi\rangle=\langle F, f\rangle=\sum\limits_{\abs{\alpha}\leq N} \frac{a_\alpha}{\alpha!}\langle F, x^\alpha\rangle$. Let $c_\alpha=\frac{\langle F, x^\alpha\rangle}{\alpha!}$, then
\begin{equation*}
\langle F, \phi\rangle=\sum\limits_{\abs{\alpha}\leq N}c_\alpha \partial^\alpha \phi(0)=\langle \sum\limits_{\abs{\alpha}\leq N} c_\alpha \partial^\alpha \delta, \phi\rangle.
\end{equation*}
Therefore, $F=\sum\limits_{\abs{\alpha}\leq N} c_\alpha \partial^\alpha \delta$.
\end{enumerate}
\section*{Problem 4}
I'll follow the steps in exercise 20 in sec 9.2.
\begin{enumerate}
\item[(1)] If $\psi\in \mathcal{S}$, which is Schwartz class, then $G*\psi\in \mathcal{S}$ for $G\in \mathcal{E}'$.
\begin{proof}
Since $G\in \mathcal{E}'$, choose $\psi\in C_c^\infty(U)$ with $\phi=1$ on $\text{supp}(G)$, and define the linear functional $H$ on $C_c^\infty(U)$ by $\langle H,\psi \rangle=\langle G,\phi\psi\rangle$. Then, it is continuous on $C_c^\infty(U)$ and unique continuous extension of $G$. Taking restriction of $H$ to $\mathcal{S}$, we can make tempered distribution $H$ and $\langle H, \psi\rangle=\langle G, \phi\psi\rangle$ for $\psi\in \mathcal{S}$. To show that $G*\psi$ is in Schwartz class, I'll compute each semi-norms in Schwartz class. For $m\in\mathbb{N}$, $\alpha$,
\begin{equation*}
\begin{split}
\norm{\partial^\alpha(G*\psi)}_{N,\alpha}&=\sup_{x\in \rr^n}(1+\abs{x})^N\abs{\partial^\alpha(G*\psi)(x)}\\
&=\sup_{x\in \rr^n}(1+\abs{x})^N\abs{G*\partial^\alpha\psi} \\
&=\sup_{x\in \rr^n}(1+\abs{x})^N\abs{\langle G, (\tau_x \widetilde{\partial^\alpha\psi})\phi\rangle} \\
&=\sup_{x\in \rr^n}(1+\abs{x})^N\abs{\langle G, (\tau_x \widetilde{\partial^\alpha\psi})\phi\rangle} \\
&\leq C\sup_{x\in \rr^n}(1+\abs{x})^N\sum\limits_{\abs{\beta}\leq M}\sup_{y\in \rr^n}(1+\abs{y})^m\abs{ \partial^\beta((\tau_x\widetilde{\partial^\alpha\psi})\phi)} \\
&\leq C\sup_{x\in \rr^n}(1+\abs{x})^N\sum\limits_{\abs{\beta}\leq M}\sup_{y\in \rr^n}(1+\abs{y})^m\abs{ \partial^\beta(\partial^\alpha\psi(x-y))\phi(y))}
\end{split}
\end{equation*}
Since $\phi\in C_c^\infty$, ...$\norm{\partial^\alpha(G*\psi)}_{N,\alpha}<\infty$ for all $N$ and $\alpha$. Therefore, $G*\psi
\in \mathcal{S}$
By the definition of the convolution,
\begin{equation*}
H*\psi(x)=\langle H, \tau_x \tilde{\psi}\rangle=\langle G, (\tau_x\tilde{\psi})\phi\rangle
\end{equation*}
for $\psi\in \mathcal{S}$. So, we can define $G*\psi(x)=\langle G, (\tau_x\tilde{\psi})\phi)$ and get $H*\psi(x)=G*\psi(x)$...(Defined region?) Since $H\in \mathcal{S}'$ and $\psi\in \mathcal{S}$, $(H*\psi)$ is a slowly increasing $C^\infty$ function, and furthermore, $(H*\psi)\,\widehat{}=\hat{\psi}\hat{H}$:
\begin{equation*}
\langle{(H*\psi)}\,\widehat{}, \varphi\rangle=\langle H*\psi, \hat{\varphi}\rangle=\langle H, \hat{\varphi}*\tilde{\psi}\rangle=\langle H, \left(\varphi*\check{\tilde{\psi}}\right)\,\widehat{}~\rangle=\langle \hat{H},\varphi*\hat{\psi}\rangle=\langle\hat{\psi}\hat{H},\varphi\rangle
\end{equation*}
for $\varphi\in \mathcal{S}$. It implies $(G*\psi)\,\widehat{}=\hat{\psi}\hat{G}$.(Direct approach)...
\end{proof}
\item[(3)]
Let's define $F*G$ by
\begin{equation*}
\langle F*G,\psi\rangle=\langle F,\tilde{G}*\psi\rangle.
\end{equation*}
for $\psi\in \mathcal{S}$. Since $\tilde{G}\in \mathcal{E}'$, $\tilde{G}*\psi\in \mathcal{S}$, and $\langle F,\tilde{G}*\psi\rangle$ is well-defined since there is continuous extension of $G$ to $\mathcal{S}$ and $F$ is tempered distribution, i.e., if $\psi_j\rightarrow \psi$ in $\mathcal{S}$, $\tilde{G}*\psi_j=\tilde{H}*\psi_j$ for some $\tilde{H}\in \mathcal{S}'$ since $\tilde{G}\in \mathcal{E}'$, and $\tilde{H}*\psi_j\rightarrow \tilde{H}*\psi=\tilde{G}*\psi$ as $j\rightarrow \infty$, so $\tilde{G}*\psi_j\rightarrow \tilde{G}*\psi$. Thus, $\langle F,\tilde{G}*\psi_j\rangle\rightarrow \langle F,\tilde{G}*\psi\rangle$, implying $F*G$ is continuous linear functional on $S$.

Since $F*G$ is tempered distribution, we can define Fourier transform of it, then
\begin{equation*}
\langle(F*G)\,\widehat{}, \psi\rangle=\langle(F*G), \hat{\psi}\rangle=\langle F, \tilde{G}*\hat{\psi}\rangle.
\end{equation*}
Also,
\begin{equation*}
\langle \tilde{G}*\hat{\psi}, \varphi\rangle=\langle \tilde{G}, \varphi*\tilde{\hat{\psi}}\rangle=\langle \tilde{G}, \varphi*\check{\psi}\rangle=\langle \tilde{G}, (\hat{\varphi}\psi)\check{}\,\rangle=\langle \hat{G}, \hat{\varphi}\psi\rangle=\langle (\hat{G}\psi)\,\widehat~~, \varphi\rangle
\end{equation*}
Therefore,
\begin{equation*}
\langle F, \tilde{G}*\hat{\psi}\rangle=\langle F(\hat{G}\psi)\,\widehat~\rangle=\langle \hat{F},\hat{G}\psi\rangle=\langle\hat{F}\hat{G},\psi\rangle
\end{equation*}
since we can view $\hat{G}$ a slowly increasing $C^\infty$ function as $G\in \mathcal{E}'$.
\end{enumerate}
\section*{Problem 5}
\begin{enumerate}
\item[(1)] Without $0$, $(x^2+y^2)^{-1}$ is continuous function, and it is locally integrable function on $\mathbb{C}\setminus\{0\}$, which is second countable. Therefore, $(x^2+y^2)^{-1}dxdy$ is a Radon measure.
I'll show the left and right-invariant. For $f\in C_c^+$, $z_0=x_0+y_0i\in \mathbb{C}\setminus\{0\}$
\begin{equation*}
\int_{\mathbb{C}\setminus\{0\}} \frac{f((x_0+y_0i)(x+iy))}{x^2+y^2}dxdy=\int_{\mathbb{C}\setminus\{0\}} \frac{f((x_0x-y_0y)+(x_0y+y_0x)i)}{x^2+y^2}dxdy.
\end{equation*}
Let $s=x_0x-y_0y$, $t=x_0y+y_0x$. Then, $dsdt=(x_0^2+y_0^2)dxdy$ and $\frac{x_0s+y_0 t}{x_0^2+y_0^2}=x$, $\frac{-y_0s+x_0t}{x_0^2+y_0^2}=y$. Since it is linear transformation with full rank, it maps $\mathbb{C}\setminus\{0\}$ to $\mathbb{C}\setminus\{0\}$.
\begin{equation*}
\begin{split}
\int_{\mathbb{C}\setminus\{0\}} \frac{f((x_0x-y_0y)+(x_0y+y_0x)i)}{x^2+y^2}dxdy&=\int_{\mathbb{C}\setminus\{0\}} \frac{f(s+ti)}{\left(\frac{x_0s+y_0 t}{x_0^2+y_0^2}\right)^2+\left(\frac{-y_0s+x_0t}{x_0^2+y_0^2}\right)^2}(x_0^2+y_0^2)^{-1}dsdt \\
&=\int_{\mathbb{C}\setminus\{0\}} \frac{f(s+ti)}{s^2+t^2}dsdt
\end{split}
\end{equation*}
It proves left-invariant. Since $\mathbb{C}\setminus\{0\}$ is Abelian group, left-invariant implies right-invariant. Hence, it is Haar measure.


\item[(2)]Again, $x^{-1}$, $x^{-2}$ is continuous function without $x=0$, and it is locally integrable function on $\{(x,y)\in\rr^2\mid x>0\}$. Therefore, $x^{-1}dxdy$ and $x^{-2}dxdy$ are Radon measure on $G$, which is homeomorphic to $\{(x,y)\in\rr^2\mid x>0\}$. I'll show left and right-invariant. Let
\begin{equation*}
A=\begin{bmatrix}
a & b \\
0 & 1
\end{bmatrix}
\end{equation*}
\begin{equation*}
\int_G f(\left(\begin{bmatrix}
a & b \\
0 & 1
\end{bmatrix}\begin{bmatrix}
x & y \\
0 & 1
\end{bmatrix}\right)x^{-2}dxdy=\int_G f(\left(\begin{bmatrix}
ax & ay+b \\
0 & 1
\end{bmatrix}\right)x^{-2}dxdy.
\end{equation*}
Let $s=ax$, $t=ay+b$, then $dsdt=a^2dxdy$ and the integration region does not change since it is just linear transformation.
\begin{equation*}
\int_G f(\left(\begin{bmatrix}
ax & ay+b \\
0 & 1
\end{bmatrix}\right)x^{-2}dxdy=\int_G f(\left(\begin{bmatrix}
s & t \\
0 & 1
\end{bmatrix}\right)(s/a)^{-2}a^{-2}dsdt=\int_G f(\left(\begin{bmatrix}
s & t \\
0 & 1
\end{bmatrix}\right)s^{-2}dsdt.
\end{equation*}
For right-invariant:
\begin{equation*}
\int_G f(\left(\begin{bmatrix}
x & y \\
0 & 1
\end{bmatrix}\begin{bmatrix}
a & b \\
0 & 1
\end{bmatrix}\right)x^{-1}dxdy=\int_G f(\left(\begin{bmatrix}
ax & bx+y \\
0 & 1
\end{bmatrix}\right)x^{-1}dxdy.
\end{equation*}
Let $s=ax$, $t=bx+y$, then $dsdt=adxdy$. The integration region does not change since it is also a linear transformation.
\begin{equation*}
\int_G f(\left(\begin{bmatrix}
ax & bx+y \\
0 & 1
\end{bmatrix}\right)x^{-1}dxdy=\int_G f(\left(\begin{bmatrix}
s & t \\
0 & 1
\end{bmatrix}\right)(s/a)^{-1}a^{-1}dsdt=\int_G f(\left(\begin{bmatrix}
s & t \\
0 & 1
\end{bmatrix}\right)s^{-1}dsdt.
\end{equation*}
Therefore, they are Haar measure.
\end{enumerate}
\end{document}